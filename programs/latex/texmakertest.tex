\documentclass{article}
\usepackage[dvipsnames]{xcolor}
\begin{document}
This example shows how to use the \texttt{xcolor} package 
to change the colour of \LaTeX{} page elements.

\begin{itemize}
\color{ForestGreen}
\item First item
\item Second item
\end{itemize}

\noindent
{\color{RubineRed} \rule{\linewidth}{0.5mm}}

The background colour of text can also be \textcolor{red}{easily} set. For 
instance, you can change use an \colorbox{BurntOrange}{orange background} and then continue typing.
I'm writing to demonstrate use of automatically-generated footnote markers\footnote{\begin{footnotesize}Automatically generated footnote markers work fine!\end{footnotesize}} and footnotes which use a marker value provided to the command\footnote[42]{\begin{footnotesize}

is that the answer to everything?\end{footnotesize}}. 

Now, I will use another automatically-generated footnote marker\footnote{\begin{footnotesize}Now, footnote markers are 1, 42, but then back to 2? That will be confusing if the automatically-generated number also reaches 42!\end{footnotesize}}.
\end{document}